\pdfminorversion=6
\documentclass[8pt, a4paper, landscape, xcolor=dvipsnames]{extarticle}
\pagestyle{empty} % Keine Seitennummern

% Verwendete Pakete
    \usepackage{chemfig}
	\usepackage[utf8]{inputenc}
	\usepackage[top=0.7cm, bottom=0.9cm, left=0.65 cm, right=0.65 cm, ]{geometry}
	\usepackage{amsmath}
	\usepackage{amsfonts}
	\usepackage{lmodern}
	\usepackage{graphicx}
	\setlength{\parindent}{0pt}
	\usepackage[normalem]{ulem}
	\usepackage{enumitem}
	\usepackage{mathabx}
	\usepackage{colortbl}
	\usepackage[ngerman]{babel}
	\usepackage{mathtools}
	\usepackage{wallpaper}
	\usepackage{changepage}
	\usepackage{tikz}
	\usepackage{tabularx}
	\usepackage{tcolorbox}
	\usepackage{lipsum}
	\usepackage{multicol}
	\usepackage{letltxmacro}
	\usepackage{tabularx}
	\usepackage{chemformula}
	\usepackage[version=4]{mhchem}
	\usepackage[x-4]{pdfx}
	%\usepackage[dvipsnames]{xcolor}


	\definecolor{Spezi1}{HTML}{512D6D}%lila
	\definecolor{Spezi2}{HTML}{ED680B}%Orange	
	\definecolor{Spezi3}{HTML}{FBB900}%Gelb


	%\definecolor{Spezi1}{HTML}{009EE3}%cyan
	%\definecolor{Spezi2}{HTML}{E5007D}%Orange	
	%\definecolor{Spezi3}{HTML}{FFED00}%Gelb
	

% Spalteneinstellungen

	\setlength\columnsep{3mm}
	\setlength{\columnseprule}{0pt}
	
	
	\setlength{\tabcolsep}{12pt} % Default value: 6pt
    \renewcommand{\arraystretch}{1.8}
            
    % Horizontale Punkte
    \LetLtxMacro\orgddots\ddots
	\DeclareRobustCommand\vdots{%
	  \mathpalette\@vdots{}%
	}
	
	\DeclareRobustCommand\ddots{%
		\mathinner{%
		   \mathpalette\@ddots{}%
		   \mkern\thinmuskip
		}%
	}

	\makeatother
	
	% Schriftart
	\renewcommand{\familydefault}{\sfdefault}
	\newcommand{\cb}{\vfill\null\columnbreak}
	% Dokument-Info Block	

	\newcommand{\DocumentInfo}[1]{
	\begin{tcolorbox}[
		arc=0mm,
		colback=Spezi1,
		colframe=white,
		bottomrule = 0 mm,
		toprule = 0 mm,
		leftrule = 0 mm,
		rightrule = 0 mm,
		valign=center,
		left=0.5mm,
		top= 0.3 mm,
		bottom= 0.3 mm,
		fontupper=\color{white},
		before skip = 0mm,
		leftright skip = -0.5mm,
		after skip = 0 mm]
		\large
	\textbf{#1}	
	\end{tcolorbox}
	}
	
	% Überschrift
	\renewcommand{\section}[1]{
	\begin{tcolorbox}[
			arc=0mm,
			colback=Spezi2,
			colframe=white,
			bottomrule = 0 mm,
			toprule = 0 mm,
			leftrule = 0 mm,
			rightrule = 0 mm,
			valign=center,
			left=0.5mm,
			top= 0.3 mm,
			bottom= 0.3 mm,
			fontupper=\color{white},
			before skip = 0mm,
			leftright skip = -0.5mm,
			after skip = 0 mm]

		\textbf{#1}
	\end{tcolorbox}
	}
		
	
	% Abschnitt	
	\renewcommand{\subsection}[2]{
	\begin{tcolorbox}[
			arc=0mm,
			colback=Spezi3,
			colframe=white,
			bottomrule = 0 mm,
			toprule = 0 mm,
			leftrule = 0 mm,
			rightrule = 0 mm,
			valign=center,
			left=0.5mm,
			top=0.1mm,
			bottom=0.1mm,
			before skip = 0mm,
			leftright skip = -0.5mm,
			after skip = 1.4 mm]
		\small \textbf{#1}
	\end{tcolorbox}
	
	\begin{adjustwidth}{0.5mm}{1mm}
		\footnotesize
		#2
		\vspace{0.5mm}
	\end{adjustwidth}
	}
	
	% Weisser Balken zwischen Abschnitten
	\newcommand{\WhiteSpace}[0]{
	\begin{tcolorbox}[
			arc=0mm,
			colback=white,
			colframe=white,
			bottomrule = 0 mm,
			toprule = 0 mm,
			leftrule = 0 mm,
			rightrule = 0 mm,
			valign=center,
			left=0.5mm,
			top= -0.4 mm,
			bottom= -0.4 mm,
			fontupper=\color{white},
			before skip = 0mm,
			leftright skip = -0.5mm,
			after skip = 0 mm]
	\end{tcolorbox}
	}
	
% Hintergrundbild (graue Spalten)

	\CenterWallPaper{1}{Setup/BackgroundLinien.pdf}

% TabularX Zeug (Paket für Tabellen)

	\newcolumntype{C}[1]{>{\centering\arraybackslash}p{#1}}

% TikZ Zeug (Paket für Vektorgraphiken)

	\usetikzlibrary{decorations.pathreplacing,calc}

	\newcommand{\tikzmark}[2][-3pt]{\tikz[remember picture, overlay, baseline=-0.5ex]\node[#1](#2){};}
	
	\tikzset{brace/.style={decorate, decoration={brace}},
	 brace mirrored/.style={decorate, decoration={brace,mirror}},
	}
	
	\newcounter{brace}
	\setcounter{brace}{0}
	\newcommand{\drawbrace}[3][brace]{%
	 \refstepcounter{brace}
	 \tikz[remember picture, overlay]\draw[#1] (#2.center)--(#3.center)node[pos=0.5, name=brace-\thebrace]{};
	}
	
	\newcommand{\annote}[3][]{%
	 \tikz[remember picture, overlay]\node[#1] at (#2) {#3};
	}