\subsection{5.1 Intermolekulare Wechselwirkungen}
    \begin{enumerate}
        \item \textbf{Wasserstoffbrücken} (stark)\\
            Entstehen zwischen H und O,N,F Atomen
        \item \textbf{Dipol-Dipol-Wechselwirkungen} (mittel) \\
            Entstehen durch polare Bindungen ($\Delta EN>0.5$)
        \item \textbf{Van-der-Waals-WW/Dispersionskräfte} (schwach)\\
            Entstehen durch temporäre Fluktuationen der Elektronen $\rightarrow$ temporärer Dipol,
            \textbf{Gibts es immer}.Grosse und lange Moleküle haben die stärksten Dispersionskräfte.
    \end{enumerate}

\subsection{5.2 Flüssigkeiten}
    \textbf{Gefrierpunktserniedrigung: } $\Delta T_f=K_fm$\\
    $K_f=$ Kryoskopische Konst.     $m=$ Molalität $\left[\frac{mol}{kg}\right]$\\
    Je tiefer die \textbf{Viskosität}, desto grösser die Mobilität der Moleküle. Viskosität proportional
    zur Stärke der WW. Je höher die Viskosität, desto dickflüssiger. 

\subsection{5.3 Ideale Gase}
    \begin{itemize}
        \item Wir machen 2 Annahmen:
        \begin{itemize}
            \item Gasteilchen wechselwirken nicht.
            \item Gasteilchen haben kein Volumen.
        \end{itemize}
        \item Ideales Gasgesetz: $pV = nRT=N kT$
        \item Dichte $\rho = M\frac{n}{V}=M\frac{p}{RT}$
        \item R ist die universelle Gaskonstante.\vspace*{1mm}
        \item Quadr. Mittelwert der Geschwindigkeit der Gasmoleküle:\\ $u_{rms}=\sqrt{3RT/M}$
        \item $M\left[ gmol^{-1} \right]$, $d\left[ gL^{-1} \right], V\left[L\right]$
    \end{itemize}
    \vspace{1mm}
    Bei hohen Drücken verhalten sich Gase nicht mehr ideal $\rightarrow$  korrigierte
    ideale Gasgleichung:\\\quad\quad$(p+\frac{n^2A}{V^2})(V-nb)=nRT$\\
    \textbf{Partialdruck:} Der Partialdruck ist der Anteil eines Gases am Druck des betrachteten Gasgemisches. Partialdrücke einer Gasmischung sind immer kleiner als der Gesamtdruck.\vspace{0.5mm}\\
    $p_i=n_i\cdot\frac{RT}{V}$ Gesamtdruck $=\Sigma$ aller Partialdrücke



\subsection{5.4 Osmotischer Druck}
    Der Druck der benötigt würde, um Fluss von Lösungsmittelteilchen zu unterdrücken heisst
    osmotischer Druck.
    \begin{equation*}
        \Pi = \left(\frac{n}{V}\right)RT=MRT \text{\quad} M = \text{Molarität}
    \end{equation*}