\subsection{5.2 Fluids}
    \textbf{Colligative Properties: }Changes depend on amount of solute added, but not which solute.
    $$
    \text{Clausius-Clapeyron} \left\{
        \begin{array}{ll}
            \ln(P_{vap}) = - \frac{\Delta H_{vap}}{RT} + C\\
            \ln(\frac{P_1}{P_2}) = \frac{\Delta H_{vap}}{R} \left( \frac{1}{T_2} - \frac{1}{T_1} \right)
        \end{array}
        \right.
    $$\\*

    % \begin{align*}
    %     \ln(P_{vap}) =& - \frac{\Delta H_{vap}}{RT} + C\\
    %     \ln(\frac{P_1}{P_2}) =& \frac{\Delta H_{vap}}{R} \left( \frac{1}{T_2} - \frac{1}{T_1} \right)
    % \end{align*}

    \textbf{Boiling-Point Elevation:}
    \begin{align*}
        \Delta T_b & = T_b (\textrm{solution}) - T_b (\textrm{solvent}) = iK_b m\\
        m & = \text{molality of solute}\\
        K_b & = \text{molal bp elevation constant (solvent)}\\
        i & = \text{van't Hoff factor}\\
        & = 1 \text{ for non-electrolytes}\\
        & = \text{Number of ions produced for electrolytes.\ e.g 2 for NaCl}
    \end{align*}
    
    \textbf{Vapor-Pressure Lowering: }
        \mathbox{
            P_\text{vap}^\text{sol} = X_\text{solvent} * P_\text{vap}^\text{pure}
        }
    
        Raoult's law $-$ Solution is an ideal solution. All intermolecular interactions are identical.\\

    \textbf{Freezing-Point Depression: }
        \begin{align*}
            \Delta T_f & = T_f (solution) - T_f (solvent) = -iK_{f}m\\
            m & = \text{molality of solute} = (\textrm{moles of solute}) / (\textrm{kg of solvent})\\
            K_f & = \text{molal fp depression constant (solvent)}\\
            i & = \text{van't Hoff factor}
        \end{align*}